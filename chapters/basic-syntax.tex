\chapter{Basic syntax}

Mit\"ankiele is a topic-prominent language with Austronesian alignment. That is,
the focus and topic are distinguished, and the semantic roles of the
participants in each clause are determined by a voice marked on the verb.
Mit\"ankiele marks the focus by having it follow the verb. The other elements in
the sentence are the topic, and more often than not follow the verb-focus
complex. The first of these elements (the ``primary topic'') is always assumed
to be the actor, except for in actor focus constructions with transitive verbs,
in which case it's assumed to be the patient. The rest of the elements may
relate to the verb in any of a number of ways: heir, friend, tool, end, place,
beginning, why, what, and how. For each of these roles, there exists a case
suffix that can optionally mark it explicitly, and a voice that marks the focus
as such. The focus is never marked for case.

If a verb is unmarked for voice, it is considered part of the focus, and takes a
default voice depending on the verb. Various types of verbs have various default
voices; broadly, intransitive verbs have doer focus, and transitive verbs have
victim focus. We can demonstrate intransitive verbs with \mk{saŋ\'ca} `stand':

\pex<intransitive>
\a\begingl
\gla Saŋ\'ca rep\"a mai\v{d}a. //
\glb stand   fox    forest //
\glft `The fox stands in the forest.' //
\endgl
\a\begingl
\gla Mai\v{d}a saŋ\'ca rep\"a. //
\glb forest    stand   fox //
\glft \textit{id.} //
\endgl
\xe

\noindent The doer must follow the verb, but other elements in the topic are
permitted to move about. \mk{mai\v{d}a} can be inferred to be locative from
context, but it can also be marked explicitly:

\agl Saŋ\'ca rep\"a mai\v{d}a -na. //
\bgl stand   fox    forest    -{\sc loc} //
\tgl `The fox stands in a forest.' //\endgl

\noindent The structure of individual elements is such that adjective-like forms
always come first, followed by pronoun clitics, followed by the noun-like form,
followed by the postposition, if any. Postpositions tend to agree with their
objects in case, although this is not strictly necessary (see
\S\ref{s:unifying}):

\ex
\begingl
\gla Saŋ\'ca wire  -p\v{c}i mi- rep\"a l\"upi -p\v{c}i mai\v{d}a -na
  keske  -n\"a. //
\glb stand   blood -color   1=  fox    leaf   -color   forest    -{\sc loc}
  center -{\sc loc} //
\glft `I, the red fox, stand in the middle of a green forest.' //
\endgl
\xe

\noindent An analogous example for a transitive verb finds the patient as the
focus, unless the doer focus suffix \mk{-l} is applied to the verb:

\pex
\a\begingl
\gla W\"apti      \'nomali mi. //
\glb {catch prey} rabbit   1 //
\glft `I catch the rabbit.' //
\endgl
\a\begingl
\gla W\"apti      -l        mi \'nomali. //
\glb {catch prey} -{\sc df} 1  rabbit //
\glft `I catch rabbits.' //
\endgl
\xe

\noindent Because \mk{w\"apti} is durative by default and \mk{\'nomali} is not
marked for number, the meaning of the sentence when \mk{\'nomali} is not the
focus shifts to the doer being generally involved in the catching of rabbits.
There are various ways to make the action more specific; one way is to clarify
the number of \mk{\'nomali} as singular, or clarify it as definite with a
pronoun:

\pex
\a\begingl
\gla W\"apti      -l        mi \"u\'n-   \'nem\"ali. //
\glb {catch prey} -{\sc df} 1  {\sc sg}- rabbit //
\glft `I catch a rabbit.' //
\endgl
\a\begingl
\gla W\"apti      -l        mi e-   \'nomali. //
\glb {catch prey} -{\sc df} 1  the= rabbit //
\glft `I catch the (recently referenced) rabbit.' //
\endgl
\xe

\noindent Another way is to change the aspect of the verb to pointive or
flashitive, causing the action to become atomic instead of durative:

\ex
\begingl
\gla W\"apti      -ki         -l        mi \'nomali. //
\glb {catch prey} -{\sc pntv} -{\sc df} 1  rabbit //
\glft `I will catch a rabbit.' //
\endgl
\xe

\noindent Perfective items don't make much sense in the present, so in the
default, nonpast tense, they are interpreted as future.

Returning to Ex.~\getref{intransitive}, we can demonstrate a similar effect with
a place focus verb:

\ex
\begingl
\gla Saŋ\'ca -\'n\'si  mai\v{d}a rep\"a. //
\glb stand   -{\sc pf} forest    fox //
\glft `The fox stands in the forest.' //
\endgl
\xe

\noindent Note that it would be ungrammatical to use the locative case with
\mk{mai\v{d}a} here. The focus never takes cases.

If one wants to front the focus, then one can replace the segment after the verb
with the dummy pronoun \mk{i}, which indicates the introduction of a new focus,
and put the focus before the verb. Since verb roots end in vowels, they usually
acquire a voice suffix in this situation to disambiguate them from the \mk{i}.

\ex
\begingl
\gla Rep\"a saŋ\'ca -\'c      i    mai\v{d}a. //
\glb fox    stand   -{\sc df} this forest //
\glft `It is the fox who stands in a forest.' //
\endgl
\xe

\noindent In these sentences, everything preceding the verb is considered the
focus.

\section{Linking}\label{s:linking}
\section{Verb serialization}
\section{The verbless interpretation}
