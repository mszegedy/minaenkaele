\chapter{Phonology and orthography}
\section{Inventory and allophony}
\subsection{Consonants}
\begin{fullwidth}\ipa
  \tabcolsep=1.8em
  \begin{tabular}{
      @{\hskip 1em}c
      l@{\hskip 1em}l
      l@{\hskip 1em}l
      l@{\hskip 1em}l
      l@{\hskip 1em}l
      @{\hskip 1em}}
    \toprule\midrule
    Labial
    & \multicolumn{2}{c}{\makebox[0pt]{Alveolar}}
    & \multicolumn{2}{c}{\makebox[0pt]{Postalveolar}}
    & \multicolumn{2}{c}{\makebox[0pt]{Prepalatal}}
    & \multicolumn{2}{c}{\makebox[0pt]{Velar}} \\\midrule
    m & n &&&& ȵ & \ortho{\'n} & ŋ & \\
    p & t && t͡ʃ & \ortho{\v{c}} & t͡ɕ & \ortho{\'c}& k & \\
    & s && ʃ & \ortho{\v{s}} & (ɕ) & \ortho{\'s} \\
    w & ð̞ l r & \ortho{d l r} &&& ð̞ʲ j & \ortho{đ j} & ɣ\,\~{}\,ɰ & \ortho{g}
    \\\bottomrule
  \end{tabular}
\end{fullwidth}
\vspace{1em}

\noindent Min\"ank\"ale's consonants don't change much based on where they are.
The stops and affricates \mbox{/\upa{p t t͡ʃ t͡ɕ k}/} can occur as geminates
anywhere in the middle of a word, but not at the edges. /\upa{ɣ}/ is realized as
[\upa{ɰ}] between vowels and before consonants, and [\upa{ɣ}] elsewhere.
/\upa{t͡ɕ}/ becomes [\upa{ɕ}] at the end of syllables, where it's spelled
\ortho{\'s}. Between vowels, it wavers between either, as [\upa{t͡ɕ\,\~{}\,ɕ}],
and is still spelled~\ortho{\'c}. It is also marginally phonemic in a small set
of Indo-Iranian loanwords that begin with [\upa{ɕ}], such as
\mk{\'s\"a\'c\'cem\"a} and \mk{\'s\"eta}.

\subsection{Vowels}\label{s:vowels}
Min\"ank\"ale\marginnote{\ipa\normalsize\centering
  \begin{tabular}{llll}
    \toprule
    i y & \ortho{i \"u} & \multicolumn{1}{r}{u} & \\
    e && ɤ o & \ortho{\"e o} \\
    æ & \ortho{\"a} & ɑ & \ortho{a} \\\bottomrule
  \end{tabular}\\[0.5em]Initial vowels\\\vspace{1em}
  \begin{tabular}{ll}\toprule
    ə   & \ortho{e} \\
    a   & \ortho{a/\"a} \\\bottomrule
  \end{tabular}\\[0.5em]Non-initial vowels} has a system of eight vowels in the
first syllable of a stem, and only two vowels elsewhere. The initial vowel
/\upa{ɑ}/ is realized as [\upa{ɑ̟}]. The non-initial open vowel /\upa{a}/ is
realized as [\upa{æ}]~\ortho{\"a} in stems that begin with one of
/\upa{i~y~e~æ}/, and [\upa{ɑ̟}]~\ortho{a} otherwise (i.e.~in stems that begin
with /\upa{u~ɤ~o~ɑ}/).

There are certain stems that begin with /\upa{i}/ as the first vowel, but have
[\upa{ɑ̟}] for their second vowel, and thus for all subsequent instances of
/\upa{a}/. Thus, there are really \emph{two} /\upa{i}/ phonemes:
/\upa{i\subscr{1}}/, after which /\upa{a}/ is realized as [\upa{æ}], and
/\upa{i\subscr{2}}/, after which it is realized as [\upa{ɑ̟}]. The Latin and
Cyrillic orthographies distinguish these indirectly by notating the difference
between non-initial [\upa{æ}] and [\upa{ɑ̟}], while the Inuktitut syllabics
orthography does not distinguish the two. /\upa{i\subscr{2}}/ is rare in native
roots, but overwhelmingly common in loanwords from languages that do not have
[\upa{æ}], such as \mk{ipsegewlewgeja}, from Latin \textit{psychologia}.

\section{Phonotactics}
Min\"ank\"ale has four kinds of syllables, each with different rules:
stem-initial syllables, non-initial stem syllables, stem-final syllables with
consonant suffixes, and suffix syllables.

Stem-initial syllables are of the form (C)V(C). The initial consonant, if
present, may be any of Min\"ank\"ale's consonants, apart from \mk{d}, \mk{g},
\mk{ŋ}, and \mk{r}. (It can't be \mk{\'s}, either, since that's just \mk{\'c} at
the end of a syllable.) The vowel can be any of the eight different stem-initial
vowels discussed above~(\S\ref{s:vowels}). The final consonant can be anything,
so long as it forms a legal consonant cluster with the next consonant.

Non-initial stem syllables are of the form CV(C), or just CV(w) if they are
stem-final (which they usually are). In either case, the beginning consonant may
be anything, and in the non-stem-final case, so may the ending consonant.

Stem-final syllables with consonant suffixes are similar, but they can only end
in a limited set of consonants, and can also be of the form CVj\{m,\,n\}.

\deprecate

\begin{multicols}{2}\raggedcolumns
\begin{itemize}
  \item (J)\{\upa{C, Pː}\}
  \item C\{\upa{t, t͡s, t͡ʃ, t͡ɕ, r}\}
  \item \{\upa{p, k, N}\}\{F, \upa{l, ȴ}\}
  \item \{\upa{N\textsubscript{h}, F}\}P
  \item Fw
  \item \{\upa{p, t, t͡s, t͡ʃ, t͡ɕ}\}\{\upa{k, q}\}
\end{itemize}
\columnbreak
\begin{affixes}{}
  C & consonant \\
  V & vowel \\
  N & \upa{m, n, ȵ, ŋ} \\
  N\textsubscript{h} & homorganic nasal \\
  P & \upa{p, t, k, q} \\
  J & \upa{w, l, ȴ, j, r, χ} \\
  F & \upa{s, ð, ʃ, ɕ, ðʲ, ɣ, χ} \\
\end{affixes}
\end{multicols}
\noindent The differences between /\upa{n}/ and /\upa{ȵ}/, and between \upa{l}
and \upa{ȴ}, are neutralized before \{\upa{t, t͡s, t͡ʃ, t͡ɕ, r}\}, the preceding
consonant assimilating to become homorganic.

In loanwords, Mit\"ankiele adds \mk{i-} to initial consonant clusters, and
\mk{-e-} to break up illegal medial consonant clusters. Sequences of vowels are
generally broken up with \mk{-w-} and \mk{-j-}, but some vowels are turned into
approximants when legal and close enough, such as the \mk{w} and \mk{j} in
\mk{piwlewgeja}~[\upa{ˈpiwləwɣəjɑ̟}] (< Latin \textit{biologia}).

\section{Sandhi}
The main source of sandhi in Mit\"ankiele is the fact that it allows neither
consecutive vowels nor glottal stops. Instead, when there are vowels on both
sides of a word boundary, it inserts an epenthetic approximant. Before front
vowels, this approximant is [j], and before back vowels, [w]. This causes the
compound \mk{mi-e}, for example, to be realized as [\upa{mí‿\,jé̞}]. (For a
discussion of tone in compounds, see \S\ref{s:pitch-accent}.)

\section{Morphophonology}
\subsection{Suffixes}
All of Min\"ank\"ale's morphology is suffixing. There are three
morphophonological processes concerning the addition of suffixes: vowel harmonic
assimilation, final vowel elision, and intervocalic epenthesis.
For\marginnote{\normalsize%
  \begin{affixes}{Harmony rules}
    \hspace{1.3em}a  & $\to$ \"a \\
    \hspace{1.3em}o  & $\to$ e \\
    \hspace{1.3em}u  & $\to$ \"u \\
    \hspace{1.3em}uo & $\to$ ie \\
  \end{affixes}
} vowel harmonic assimilation, once a suffix is attached, its vowels assimilate
to the vowel harmony class of the root it's attached to. All prefixes in this
grammar are given as their back vowel variants; to transform them into their
front vowel variants, one simply applies the changes given to the right, for
both the short and long variants of the vowels.

Final vowel ellision occurs to \mk{-i}, and to \mk{-o/e} in un-ablauted
stems. It only occurs if the resulting combination is phonotactically
permissible. Therefore, it always happens when the ending begins with a vowel:
\mk{suŋe-en}~>~\mk{suŋen}. Conversely, it never happens when the final vowel is
preceded by a consonant cluster, or if the suffix contains a consonant cluster,
with one exception covered just below: \mk{mai\v{d}a-na}~>~\mk{mai\v{d}ana},
\mk{suŋe-lsi}~>~\mk{suŋelsi}. It also never occurs if it would create a final
consonant cluster, meaning that it doesn't occur to consonant suffixes at all:
\mk{suŋe-n}~>~\mk{suŋen}. In all other cases, one has to evaluate the potential
consonant cluster for validity: \mk{\mbox{jomi-ta}}~>~\mk{jomta}, but
\mk{jomi-ka}~>~\mk{jomika}.

In cases where the elision would create \mk{-uwC} or \mk{-ijC}, the approximant
is vocalized: \mk{puwi-na}~>~\mk{puuna}, \mk{kije-li}~>~\mk{kiili}. Similarly,
if it would double a consonant, then for consonants capable of becoming
geminates (all but central approximants) become geminates, while other
consonants lengthen the vowel behind them (if it isn't long already):
\mk{\'cele-le}~>~\mk{\'celle}. This also applies in the case that one or both of
the consonants are geminate: \mk{\mbox{\v{c}itti-ta}}~>~\mk{\v{c}itt\"a}. This
also creates the singular exception to the rule that roots in which the final
vowel is preceded by a consonant cluster never experience elision: cases in
which the ending can assimilate into a geminate with the last consonant of the
cluster in the root, such as \mk{\'c\"arke-ki}~>~\mk{\'c\"arkki}.

The final process, intervocalic epenthesis, is much simpler. It concerns
epenthetic consonants inserted when a suffix begins with a vowel, and the final
vowel of the root can't elide. Which consonant is inserted depends entirely on
the first vowel, i.e.~the root's: if it is front, then \mk{-j-} is inserted, and
if it is back, then \mk{-g-} is inserted, which is realized as [\upa{ɰ}] in this
environment. Examples are \mk{w\"ul\"u-u}~>~\mk{w\"ul\"uj\"u},
\mk{\'caka-a}~>~\mk{\'cakaga}, \mk{\mbox{kudu-en}}~>~\mk{kudugen}. Examples where the
first vowel is front and the second vowel is back do not exist, since the only
possible vowels in back vowel roots are \mk{-i} or \mk{-e}, which elide before
suffixes that begin with vowels.

\subsection{Pitch accent in compounds}\label{s:pitch-accent}
In hyphenated compounds, such as those formed with classifiers, modifying
pronouns, and numerals, the individual elements within the compound are
distinguished by pitch accent. After the first element of the compound, all
subsequent elements start with roughly the same pitch as the first syllable of
the first element, perhaps slightly lower. Additionally, in multisyllabic
non-final elements, the last syllable gets a rising tone, to help meet the high
tone that the next syllable begins with. For example, this is how the pitch
melody of \mk{kekt\"a-kutte-\"ukte e-rep\"a} 'thirteen of these foxes' is
rendered:

\begin{center}
  \begin{tikzpicture}
    \contour[contour raise=0.5cm]
            {|[10]kek |[5]tä- |[9]ku |[4]tte- |[9]ük te|[2] |[6]e- |[6]re pä|[0]}
  \end{tikzpicture} 
\end{center}
\vspace{-1em}
\noindent This may be transcribed into IPA as [\upa{kêktæ̌‿kûtːě‿jýkte̞ é‿répæ}].
Note the falling tone on \mk{kek} and \mk{ku}; this happens in the first
syllable of every polysyllabic non-final element.
\pagebreak

\undeprecate
\section{Orthography}
\subsection{Latin alphabet}
Given that acronyms are an occasional component of the language, letter names
are worth clarifying. The names of the 27 letters of the Min\"ank\"ale alphabet,
along with their ASCII representation if it differs, are:
\begin{center}
  \begin{tabular}{c@{\hskip 2pt}cc@{\hskip 2pt}cl}
    A & a &&& \mk{a} \\
    \"A & \"a & Ae & ae & \mk{\"a} \\
    \'C & \'c & Cj & cj & \mk{\'ca} \\
    \v{C} & \v{c} & Cz & cz & \mk{\v{c}a} \\
    D & d &&& \mk{\'cade} or \mk{da} \\
    Đ & đ & Dj & dj & \mk{đa} \\
    E & e &&& \mk{e} \\
    \"E & \"e & Eo & eo & \mk{\"e} \\
    G & g &&& \mk{kimele} or \mk{ga} \\
    I & i &&& \mk{i} \\
    J & j &&& \mk{je} \\
    K & k &&& \mk{ka} \\
    L & l &&& \mk{le} \\
    M & m &&& \mk{ma} \\
  \end{tabular}\hspace{1em}
  \begin{tabular}{c@{\hskip 2pt}cc@{\hskip 2pt}cl}
    N & n &&& \mk{na} \\
    \'N & \'n & Nj & nj & \mk{\'na} \\
    Ŋ & ŋ & Ng & ng & \mk{naŋe} or \mk{ŋa} \\
    O & o &&& \mk{o} \\
    P & p &&& \mk{pa} \\
    R & r &&& \mk{re} \\
    S & s &&& \mk{sa} \\
    \'S & \'s & Sj & sj & \mk{\'sa} \\
    \v{S} & \v{s} & Sz & sz & \mk{\v{s}a} \\
    T & t &&& \mk{ta} \\
    U & u &&& \mk{u} \\
    \"U & \"u & Ue & ue & \mk{\"u} \\
    W & w &&& \mk{we} \\
  \end{tabular}
\end{center}
Apart from the above-mentioned graphemes and the behavior of the phonemes they
represent, worth discussing is Min\"ank\"ale punctuation and typesetting. While
loanwords are generally rendered phonetically, it is also acceptable to render
them in their original spelling, in which case they are italicized unless they
are a proper noun, and any appended prefixes or suffixes are connected with a
colon. Acronyms likewise receive suffixes with a colon. Quotes are done with
«~and~», and nested quotes are done with ‹~and~›. Punctuation does not change in
quotations, affixes are added directly to them without any other indication,
clitics are added to them via hyphens, and they do not change the following
capitalization; in short, they act exactly like words. In ASCII, · is replaced
with~\texttt{-}, «~and~» are both replaced with~\texttt{"}, and ‹~and~› are both
replaced with~\texttt{'}. Hence:
\bigexample{%
  «El\"a SMS·mege·pelek me \textit{auditorium}:na.» monajme Jonathan:ŋ. \\
  \texttt{"Elae SMS-mege-pelek me auditorium:na." monajme Jonathan:ng.}}{%
  ``Don't be afraid to send me an SMS at the auditorium,'' I told Jonathan.}

\subsection{Cyrillic alphabet}
\begin{fullwidth}
  \tabcolsep=1.8em
  \cyrfont
  \begin{tabular}{
      l@{\hskip 1em}l
      l@{\hskip 1em}l
      l@{\hskip 1em}l
      l@{\hskip 1em}l
      l@{\hskip 1em}l
      @{\hskip 1em}}
    \toprule\midrule
    \multicolumn{2}{c}{\makebox[0pt]{\latfont Labial}}
    & \multicolumn{2}{c}{\makebox[0pt]{\latfont Alveolar}}
    & \multicolumn{2}{c}{\makebox[0pt]{\latfont Postalveolar}}
    & \multicolumn{2}{c}{\makebox[0pt]{\latfont Prepalatal}}
    & \multicolumn{2}{c}{\makebox[0pt]{\latfont Velar}} \\\midrule
    м & \ortho{m} & н     & \ortho{n}
    &&& њ   & \ortho{\'n} & ӈ & \ortho{ŋ} \\

    п & \ortho{p} & т     & \ortho{t} & тш & \ortho{\v{c}}
    &   ч   & \ortho{\'c} & к & \ortho{k} \\

      &           & с     & \ortho{s} & ш  & \ortho{\v{s}}
    &   щ   & \ortho{\'s} \\

    в & \ortho{w} & д л р & \ortho{d l r}
    &&& з й & \ortho{đ j} & г & \ortho{g} \\\bottomrule
  \end{tabular}
\end{fullwidth}
\vspace{1em}

\noindent Min\"ank\"ale's\marginnote{\cyrfont\normalsize\centering
  \begin{tabular}{llll}
    \toprule
    и ы & \ortho{i \"u}
    & \multicolumn{1}{r}{у} & \multicolumn{1}{r}{\ortho{u}} \\
    е   & \ortho{e}     & ъ о & \ortho{\"e o} \\
    э   & \ortho{\"a}   & а   & \ortho{a} \\\bottomrule
  \end{tabular}\\[0.5em]{\latfont Initial vowels}\\\vspace{1em}
  \begin{tabular}{llll}\toprule
    ы   & \ortho{e}     \\
    а э & \ortho{a \"a} \\\bottomrule
  \end{tabular}\\[0.5em]{\latfont Non-initial vowels}} Cyrillic orthography
contains only two letters not in the Russian alphabet, \ortho{\cyrfont њ} and
\ortho{\cyrfont ӈ}. It has a one-to-one mapping with the Latin orthography,
except for both \mk{\v{c}} and \mk{t\v{s}} corresponding to \ortho{\cyrfont тш}.
In practice, however, \mk{t\v{s}} never appears, even in loanwords. The example
concluding the previous section is written:
\bigexample{\cyrfont %
  «Елэ SMS·мегы·пелык ме auditorium:на.» монаймы Jonathan:ӈ.}{%
  ``Don't be afraid to send me an SMS at the auditorium,'' I told Jonathan.}

\subsection{Inuktitut syllabics}
\begin{fullwidth}
  \tabcolsep=1.8em
  \inufont
  \begin{tabular}{
      l@{\hskip 1em}l
      l@{\hskip 1em}l
      l@{\hskip 1em}l
      l@{\hskip 1em}l
      l@{\hskip 1em}l
      @{\hskip 1em}}
    \toprule\midrule
      \multicolumn{2}{c}{\makebox[0pt]{\latfont Labial}}
    & \multicolumn{2}{c}{\makebox[0pt]{\latfont Alveolar}}
    & \multicolumn{2}{c}{\makebox[0pt]{\latfont Postalveolar}}
    & \multicolumn{2}{c}{\makebox[0pt]{\latfont Prepalatal}}
    & \multicolumn{2}{c}{\makebox[0pt]{\latfont Velar}} \\\midrule
    ᒪ & \ortho{m} & ᓇ & \ortho{n}
    &&& ᓐᔭ & \ortho{\'n} & ᖓ & \ortho{ŋ} \\

    ᐸ & \ortho{p} & ᑕ & \ortho{t} & ᑦᖬ & \ortho{\v{c}}
    &   ᑦᔭ & \ortho{\'c/\'s} & ᑲ & \ortho{k} \\

      &           & ᓴ & \ortho{s} & ᖬ & \ortho{\v{s}} \\

    ᕙ & \ortho{w} & ᖤ ᓚ ᕋ & \ortho{d l r}
    &&& ᖦᔭ ᔭ & \ortho{đ j} & ᒐ & \ortho{g} \\\bottomrule
  \end{tabular}
\end{fullwidth}
\vspace{1em}

\noindent Min\"ank\"ale's Inuktitut syllabics orthography is its simplest and
most straightforward. It does not distinguish \mk{\'c} and \mk{\'s}, though
there is room for disambiguation, as \ortho{\inufont ᔅᔭ} is unused.
\ortho{\inufont ᖓ} has a few special cases: word-final \mk{-ŋ} is simply
\ortho{\inufont ᖕ}, and the clusters \mk{ŋg} and \mk{ŋk} are \ortho{\inufont ᙵ}
and \ortho{\inufont ᖕᑲ} respectively.

Out\marginnote{\inufont\normalsize\centering
  \begin{tabular}{llll}
    \toprule
    ᐃ ᕷ & \ortho{i \"u}
      & \multicolumn{1}{r}{ᐅ} & \multicolumn{1}{r}{\ortho{u}} \\
    ᐁ   & \ortho{e}     & ᕵ ᕴ & \ortho{\"e o} \\
    ᕹ   & \ortho{\"a}   & ᐊ   & \ortho{a} \\\bottomrule
  \end{tabular}\\[0.5em]{\latfont Initial vowels}\\\vspace{1em}
  \begin{tabular}{llll}\toprule
    ᐃ & \ortho{e}     & ᐅ & \ortho{ew} \\
    ᐊ & \ortho{a/\"a} & ᐁ & \ortho{aw} \\\bottomrule
  \end{tabular}\\[0.5em]{\latfont Non-initial vowels}} of the eight stem-initial
vowels, \mbox{\mk{a i u e}} are represented as is usual for Inuktitut syllabics,
while \mbox{\mk{\"a \"e \"u o}} are associated with a modifying \ortho{\inufont
  ᕻ}. Stems beginning with vowels represent it as the full letter
\ortho{\inufont ᕹ}, while stems beginning with consonants use the full letter of
the consonant, followed by the modifying \ortho{\inufont ᕻ}, which precedes any
coda consonants.

The orthography does not distinguish non-initial \mk{a} and \mk{\"a}, and it
allows you to write syllable-final \mk{-w} without adding an extra letter.
\mk{a/\"a} and \mk{e} are represented with what would otherwise be \mbox{\mk{a
    i}}, while the equivalents of initial \mbox{\mk{e u}} stand for \mbox{\mk{aw
    ew}} non-stem-initially. Thus, the previously-mentioned loanword
\mk{ipsegewlewgeja} is rendered {\inufont ᐃᑉᓯᒍᓗᒋᔭ}. The example concluding the
previous sections is written:
\bigexample{\inufont %
  «ᐁᓚ SMS·ᒣᒋ·ᐯᓕᒃ ᒣ auditorium:ᓇ.» ᒣᕻᓇᔾᒥ Jonathan:ᖕ.}{%
  ``Don't be afraid to send me an SMS at the auditorium,'' I told Jonathan.}
\enlargethispage{2.5cm} \mk{Auditorium} remains unitalicized, both because it's
already obvious that it's a foreign word, and because Inuktitut syllabics fonts
rarely provide italics in the first place, Latin or not.
\pagebreak
