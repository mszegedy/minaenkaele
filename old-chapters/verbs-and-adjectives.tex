\chapter{Verbs and adjectives}
\section{Mood}
The\marginnote{%
  \begin{affixes}{Moods}
    -Ø  & is \\
    -ne & might \\
    -k  & isn't \\
  \end{affixes}} most fundamental distinction made on finite verbs in Mit\"ankiele
is mood.

\section{Focus}
The focus of a finite verb's object is determined by its case ending. See
\S\ref{s:case} for an overview of what cases are available. It is essentially a
transfer of the case ending of the object to that of the verb. The default focus
for intransitive verbs is usually the doer, and for transitive verbs, the
victim, but certain verbs have other default focuses; for example,
\mk{pele}~`fear' is, by default, start focus. In such situations, the kind of
focus will be given in the lexicon as a smallcaps abbreviation, as it will
appear in the gloss as well. (In the case of \mk{pele}, this would be
\morp{sf}.)

\section{Evidence}
These\marginnote{\normalsize%
  \begin{affixes}{Evidentials}
    -ci      & smell \\
    -la      & sight \\
    -ku      & hearing \\
    -lu      & touch, taste, other body sense \\
    -ga      & heard it from someone \\
    -nu      & heard it somewhere \\
    -gu      & a friend's opinion, not really mine \\
    -ni      & a stranger's opinion, not really mine \\
    -\v{c}a  & other declarative memory \\
    -wa      & other episodic memory \\
    -ra      & intuition, procedural memory \\
    -tta     & can't remember it properly \\
    -me      & best guess after thinking about it \\
    -pi      & can prove it \\
  \end{affixes}} evidentials occur exclusively after the ``might'' mood in \mk{-ne}.

\section{Non-finite forms in \mk{\rm -ma}\ and \mk{\rm -ni}}
\begin{affixes}{Non-finite forms}
  -ma & gerund \\ -(ma)ni & infinitive \\
\end{affixes}

\wdef{s\"ane}{play, be something that plays}
\wdef{s\"anem\"a}{the state of playing}

\bigexample{Karma s\"ane mi.}{I want something that plays.}

\bigexample{Karma s\"anem\"a mi.}{I want the state of playing.}

\bigexample{Karma s\"anem\"ani.}{I want to play. {\it lit.}\ Playing is
  desirable.}

\wdef{\"u\'c\"a}{small, be something that is small}
\wdef{\"u\'c\"am\"a}{the state of being small}

\bigexample{Karma \"u\'c\"a mi.}{I want something small.}

\bigexample{Karma \"u\'c\"am\"a mi.}{I want the state of being small.}

\bigexample{Karma \"u\'c\"am\"ani.}{I want to be small. {\it lit.}\ Being small
  is desirable.}

\wdef{rep\"a}{fox, be a fox}
\wdef{rep\"am\"a}{the state of being a fox}

\bigexample{Karma rep\"a mi.}{I want a fox.}

\bigexample{Karma rep\"am\"a mi.}{I want the state of being a fox.}

\bigexample{Karma rep\"am\"ani.}{I want to be a fox. {\it lit.}\ Being a fox is
  desirable.}

\noindent But we can also apply \mk{-ni} to bare stems, in which case the stem
itself is interpreted as a state of being. To Mit\"ankiele speakers, this means
that it is interpreted as, ``the state of having \ortho{original meaning} exist
for you'', or perhaps, ``the state of experiencing the existence of
\ortho{original meaning}''. Thus:

\bigexample{Karma s\"aneni.}{I want to experience the existence of something
  that plays.}

\bigexample{Karma \"u\'c\"ani.}{I want to experience the existence of
  something small.}

\bigexample{Karma rep\"ani.}{I want to experience the existence of a fox.}

\noindent It is ambiguous whether ``existence'' here refers to the existence of
a particular instance of the root, or the existence of \emph{any} instances of
the root within the universe at all, but the latter is a more likely
interpretation, since the former is just an overly elaborate way of saying that
you want an instance of the root.

\section{Auxiliary verbs}
\subsection{Taking both gerunds and infinitives}
\begin{affixes}{Auxiliary verbs that take both forms}
  e         & doesn't \\
  k\"aske   & (command) \\
  \"al\"a   & (neg.~command) \\
  ??        & let's \\
  \"alw\"a  & can \\
  ma\v{c}ta & can physically \\
  ??        & stop \\
  ??        & be tired of \\
  ??        & must \\
  ??        & should \\
  ??        & should (but isn't) \\
  kelke     & have to \\
  ??        & intend \\
  ??        & be good at \\
\end{affixes}

\subsection{Taking only gerunds}
\begin{affixes}{Aux.~verbs that only take gerunds}
  karma       & want \\
  soŋe        & wish \\
  \"a\v{c}\"a & like \\
  pele        & fear \\
  n\"ake      & seem \\
  tale        & pretend \\
  tulek       & will in a moment \\
  e\v{s}te    & finish \\
\end{affixes}
