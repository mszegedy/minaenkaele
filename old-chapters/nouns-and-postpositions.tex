\chapter{Nouns and postpositions}
\section{Inflection}
\subsection{Number}
\begin{affixes}{Numbers}
  -sa     & singular \\
  -ka     & dual \\
  -t/-ja- & plural \\
  we\'n-  & all \\
\end{affixes}
\subsection{Case}\label{s:case}
\begin{affixes}{Cases}
  -m      & subject (doer or victim) \\
  -\'n    & heir \\
  -s      & friend \\
  -\v{c}  & tool \\
  -da     & start \\
  -na     & place \\
  -ŋ      & end \\
  -\v{s}  & reason \\
  -nen    & context \\
\end{affixes}
\subsection{Classifiers}
\subsection{Derivation}
\begin{affixes}{Noun-forming suffixes from finite roots}
  -\'ci     & smol \\
  -mi       & a specific kind of \\
  -mimi     & favorite \\
  -sta      & a bunch of, the family of \\
  -jka      & a part of \\
  -jkaikka  & a fragment of \\
  -liwiika  & a container of \\
  -kka      & -ish, a ripoff of \\
  -a\'si    & hunter, herder, or farmer of \\
  -ule      & interval of, unit of \\
  -ni       & existence of \\
\end{affixes}
% other potential additions:
%   one who is good at X
%   sound of X
%   smell of X
\section{Postpositions}
